\usepackage{etex}
\usepackage[utf8]{inputenc}
\usepackage{csquotes}

\makeatletter

\usepackage{multicol}
\usepackage{etoolbox}
\usepackage{silence}
\usepackage[%
backend=biber,
style=numeric-comp,
sorting=none,
sortcites=true,
block=none,
indexing=false,
citereset=none,
isbn=false,
url=true,
doi=false,
natbib=true,
maxbibnames=99
]{biblatex}

\WarningFilter{biblatex}{Patching footnotes failed}

\renewcommand{\bibfont}{\normalfont\scriptsize}
\newcommand{\BibResourceGlobal}{../../../library.bib}
\newcommand{\BibResourceLocal}{refs.bib}
\IfFileExists{\BibResourceGlobal}
{\newcommand{\BibResource}{\BibResourceGlobal}}
{\newcommand{\BibResource}{\BibResourceLocal}}
\addbibresource{\BibResource}

\usepackage[title, titletoc]{appendix}
\usepackage[T1]{fontenc}
\usepackage{lmodern}
\usepackage{textcomp}

\usepackage{algorithm}
\usepackage{algpseudocode}
\newcommand{\Break}{\State \textbf{break} }
\algblockdefx[Loop]{Loop}{EndLoop}[1][]{\textbf{Loop} #1}{\textbf{End Loop}}

\algrenewcommand\ALG@beginalgorithmic{\footnotesize}

\usepackage{listings}

\usepackage{courier}

\lstset{frame=bt,
breaklines=true,
captionpos=b,
numbers=left,
basicstyle=\scriptsize\ttfamily,
showstringspaces=false,
numberstyle=\tiny,
upquote=true,
commentstyle=\bfseries}

\lstset{escapeinside={(*@}{@*)}}

\usepackage{amsmath}
\usepackage{esvect}
\usepackage{amsfonts}
\usepackage{amssymb}

\usepackage{graphicx}
\usepackage{mathtools}
\usepackage{tikz}
\usepackage{tikz-qtree}

\usepackage{bm}

\usepackage{subcaption}
\expandafter\def\csname ver@subfig.sty\endcsname{}

\newenvironment{myalignat}[1]{%
\setlength{\abovedisplayskip}{-.7\baselineskip}%
\setlength{\abovedisplayshortskip}{\abovedisplayskip}%
\start@align\z@\st@rredtrue#1
}%
{\endalign}

\DeclareMathOperator*{\argmin}{arg\,min}
\DeclareMathOperator*{\argmax}{arg\,max}
\DeclareMathOperator*{\gain}{Gain}
\DeclareMathOperator*{\map}{Map}
\DeclareMathOperator*{\reduce}{Reduce}
\DeclareMathOperator*{\scan}{Scan}
\DeclareMathOperator*{\stencil}{Stencil}
\DeclareMathOperator*{\zip}{Zip}
\DeclareMathOperator*{\allpairs}{All\,Pairs}

\usepackage{pgfplots}
\pgfplotsset{compat=1.12}
\usepackage{pgfgantt}

\newganttchartelement*{mymilestone}{
mymilestone/.style={
shape=diamond,
inner sep=2pt,
draw=black,
top color=black,
bottom color=black,
}
}

\usetikzlibrary{shapes,arrows,shadows,fit,backgrounds}
\tikzstyle{decision} = [diamond,
draw,
text width=4.5em,
text badly centered,
node distance=3cm,
inner sep=0pt]
\tikzstyle{block}    = [rectangle,
draw,
text width=5em,
text centered,
node distance=3cm,
minimum height=4em,
inner sep=.2cm]
\tikzstyle{line}     = [draw, -latex']

\newcount\dirtree@lvl
\newcount\dirtree@plvl
\newcount\dirtree@clvl
\def\dirtree@growth{%
\ifnum\tikznumberofcurrentchild=1\relax
\global\advance\dirtree@plvl by 1
\expandafter\xdef\csname dirtree@p@\the\dirtree@plvl\endcsname{\the\dirtree@lvl}
\fi
\global\advance\dirtree@lvl by 1\relax
\dirtree@clvl=\dirtree@lvl
\advance\dirtree@clvl by -\csname dirtree@p@\the\dirtree@plvl\endcsname
\pgf@xa=0.33cm\relax
\pgf@ya=-\baselineskip\relax
\pgf@ya=\dirtree@clvl\pgf@ya
\pgftransformshift{\pgfqpoint{\the\pgf@xa}{\the\pgf@ya}}%
\ifnum\tikznumberofcurrentchild=\tikznumberofchildren
\global\advance\dirtree@plvl by -1
\fi
}
\tikzset{
dirtree/.style={
growth function=\dirtree@growth,
every node/.style={anchor=north},
every child node/.style={anchor=west},
edge from parent path={(\tikzparentnode\tikzparentanchor) |- (\tikzchildnode\tikzchildanchor)}
}
}

\usepackage[underline=false]{pgf-umlsd}

\usepackage{svg}
\graphicspath{{img/}}

\usepackage{booktabs}
\usepackage{tabularx}
\usepackage{longtable}
\usepackage{array}
\newcolumntype{L}[1]{>{\raggedright\let\newline\\\arraybackslash\hspace{0pt}}m{#1}}
\newcolumntype{C}[1]{>{\centering\let\newline\\\arraybackslash\hspace{0pt}}m{#1}}
\newcolumntype{R}[1]{>{\raggedleft\let\newline\\\arraybackslash\hspace{0pt}}m{#1}}

\usepackage{fancyvrb}

\newcommand{\hr}{\noindent\makebox[\linewidth]{\rule{\textwidth}{0.2pt}}}
\newcommand{\br}{\hspace{\baselineskip}}

\newcommand*\wrapletters[1]{\wr@pletters#1\@nil}
\def\wr@pletters#1#2\@nil{#1\allowbreak\if&#2&\else\wr@pletters#2\@nil\fi}

\newcommand{\centredtitle}[1]{
\begin{center}
  \large
  \vspace{0.9cm}
  \textbf{#1}
\end{center}}

\usepackage{hyperref}
\hypersetup{pdfborder={0 0 0}}
\usepackage{gensymb}
\usepackage[english]{babel}
\usepackage{blindtext}
\usepackage{lipsum}
